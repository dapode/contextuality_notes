\documentclass[a4paper]{article}
%\documentclass[a5paper]{article}
\usepackage{microtype}
\usepackage[T1]{fontenc}
\usepackage[utf8x]{inputenc}
\usepackage{mathtools}
\usepackage{amssymb}
\usepackage{amsthm}
\usepackage{mathpple}
\usepackage{eulervm}
\usepackage{framed}
\usepackage{footnote}
\usepackage{booktabs}
\usepackage{xcolor}
\usepackage{caption}
\usepackage{subcaption}
\usepackage{braket}
%\usepackage[papersize={4.5in,6in},margin=0.5cm]{geometry} %per e-reader

\mathtoolsset{showonlyrefs}
% Colori
\definecolor{dgreen}{RGB}{0,150,0}
\definecolor{dred}{RGB}{150,0,0}

% Operatori
% --------------------
\DeclareMathOperator{\Tr}{Tr}
%\DeclareMathOperator{\det}{det}
\DeclareMathOperator{\atanh}{atanh}

% Abbreviazioni comode
% --------------------
\newcommand{\p}{\partial}
\newcommand{\abs}[1]{\lvert #1 \rvert}
\newcommand{\Real}{\mathbb{R}}
\newcommand{\Hil}{\mathcal{H}}
\newcommand{\op}[1]{\mathbf{#1}}
\newcommand{\acron}[1]{\textsc{#1}}
\newcommand{\HVT}{\acron{hvt}}
\newcommand{\QM}{\acron{qm}}
\newcommand{\I}{\mathbf{I}}
\newcommand{\red}[1]{\textcolor{dred}{#1}}
\newcommand{\green}[1]{\textcolor{dgreen}{#1}}

% Ambienti
% --------
\newtheorem{theorem}{Theorem}[section]

\theoremstyle{definition}
\newtheorem{definition}{Definition}[section]

\let\oldproof\proof
\let\oldendproof\endproof
\renewenvironment{proof}
    {
        \begin{framed} 
        \oldproof
    }
    {
        \oldendproof 
        \end{framed}
    }

\begin{document}
\title{Notes on Contextuality}
\author{Davide Poderini}
\date{}
\maketitle

\section{Introduction}
To do.
\section{Fundamentals}
Suppose we can perform on a system measurements of some observables $A_1, A_2,\ldots, A_n$,
whose outcomes are designated by $a_1, a_2, \ldots, a_n$, and suppose that not all of these
observables can be measured at once without disturbing the system.
We call the properties that can be measured together \emph{compatible}, and
\emph{non compatible} the ones that cannot.

We then define as the \emph{context} of the measurement of $A_i$ the other
\emph{compatible} properties that are measured together with it.
A theory is said to be \emph{non contextual} if the predicted outcome $a_i$ of some observable
$A_i$ does not depend on the context.
If it does we say that the theory is \emph{contextual}.

In quantum theory compatible measurements are represented by commuting operators
on a Hilbert space $\Hil$. 
In this section we can limit ourselves to consider only dichotomous
measurements, which are represented by projectors in \QM{}.
Compatible observables in this case are equivalent to orthogonal projectors.

In \QM{} contextuality is said to be \emph{not explicit}. This means that the outcome of a measure of an
observable $A_i$ may depend on the context, but the probability for that single
observable $p(A_i = a_i)$, does not.
This is also called the \emph{no-disturbance} or \emph{no-signaling} principle.

Nonetheless, as we will show in this section, a non-contextual \HVT{} (hidden
variable theory) cannot reproduce the prediction of \QM{}.

\subsection{Joint probability}
\label{sec:joint_prob}
The problem of the existence of a non-contextual \HVT{} can be better
expressed in terms of probability distributions.
To simplify the notation we call $f(a_1,\ldots,a_k) = p(A_1=a_1, \ldots,
A_k=a_k)$ the probability that the measure of $A_1,\ldots,A_k$ gives outcomes
$a_1,\ldots,a_k$.

Quantum mechanics gives us only the joint probability distribution for
\emph{compatible} observables.
So if $\{A_1, A_2\}$ are compatible observables, but $\{A_1, A_2, A_3\}$ are
not, \QM{} gives us only $f(a_1, a_2)$, without any information about an
hypothetical $f(a_1, a_2, a_3)$.

A non-contextual \HVT{} instead, the presence of a global joint probability distribution for all
observables $f(a_1, \ldots, a_n)$ is assured by the fact that every property of
the system must have a definite value, and the uncertainty is only given by our
ignorance about the initial conditions.
From that global joint distribution one should be able to derive the quantum
mechanical ones for compatible observables $f(a_1, \ldots, a_k)$, as marginals.

The possibility of the existence of a joint probability distribution depends on
the compatibility relations between the observables.
To better visualize them, we can construct a graph where every vertex represents
an observable and two vertex are connected if the corresponding observables are
compatible, some examples are shown in figure \ref{fig:jointprob_graphs}.
\begin{figure}[h]
    \begin{subfigure}{.3\textwidth}
        \center
        \includegraphics[width=.8\textwidth]{prob_graph1.pdf}
        \caption{}
        \label{fig:jointprob_graph1}
    \end{subfigure}
    \begin{subfigure}{.3\textwidth}
        \center
        \includegraphics[width=.7\textwidth]{CHSH_obs.pdf}
        \caption{}
        \label{fig:jointprob_CHSH}
    \end{subfigure}
    \begin{subfigure}{.3\textwidth}
        \center
        \includegraphics[width=\textwidth]{KCBS_obs.pdf}
        \caption{}
        \label{fig:jointprob_KCBS}
    \end{subfigure}
    \caption{Examples of compatibility graphs}
    \label{fig:jointprob_graphs}
\end{figure}

A sufficient condition for the existence of a joint probability
distribution is that the corresponding graph should not contain \emph{n-cycles} with $n>3$.
\footnote{The definition of \emph{n-cycle}, and other terms related to graph
theory, is given in the appendix \ref{sec:graph}.}
This statement is easily proved by constructing the joint probability for such a
graph.
\begin{proof}
If there are only 3-cycles, it is always possible to construct a global joint
probability distribution from the joint distributions of compatible observables
as follows:
first multiply together the joint distribution for every 3-cycle 
\footnote{A 3-cycle is also a clique of three mutually compatible observable, so a
joint distribution for all three variables is experimentally accessible.}
$f(a_i, a_j,a_k)$, then multiply those by the joint distributions for all the other edges,
and if an observable is present in more than one distribution divide by
$f(a_l)^{n-1}$, where $n$ is the number of terms in which it appears.
For example, for the first graph in figure \ref{fig:jointprob_graphs} we have:
\begin{equation}
    f(a_1,\ldots,a_6) = \frac{f(a_1,a_2,a_3) f(a_2, a_6) f(a_3,a_4)
    f(a_4,a_5)}{f(a_2) f(a_3) f(a_4)}
    \label{eq:jointprob_graph1}
\end{equation}
from which all the joint distributions for every edge can be obtained by summing
over the other variables.
\end{proof}
So for example it is certainly possible to construct such a distribution for
graph \ref{fig:jointprob_graph1}, but not for graphs \ref{fig:jointprob_CHSH}
and \ref{fig:jointprob_KCBS}.

While this gives no information about the existence of a joint distribution for
other graphs, the next two theorems will prove that a
similar construction is impossible in general.

\subsection{Gleason's theorem}
In the course of an analysis on the axioms of quantum theory,
Gleason proved an important theorem that poses a limit to form that the
distribution of an hypothetical \HVT{} could have.

As we said in quantum theory every yes-no test about a system can be represented by a
projector $\op P$ in a Hilbert space $\Hil$.
A \emph{quantum probability distribution} is a function $f(\op P)$ that assign to every
such projector a value such that:
\begin{align}
    &f(\op P) > 0 \\
    &\sum_i f(\op P_i) = 1 & \text{for every orthogonal base} \quad \{P_i\}
    \label{eq:gleason_th_hypotesis}
\end{align}

What Gleason proved is that every quantum probability distribution $f$ on an Hilbert space
of dimension greater than three can be written as:
\begin{equation}
    f(\op P) = \Tr(\rho \op P)
    \label{eq:gleason_prob}
\end{equation}
for a suitable positive semi-definite, hermitian operator $\rho$, such that
$\Tr(\rho) = 1$.
This means that the density matrix formalism is enough to describe the most general kind of
quantum probability distribution in an Hilbert space.

\begin{proof}
    To do.
\end{proof}

With this theorem is possible to deduce quantum theory from these three
postulates:
\begin{enumerate}
    \item Dichotomous tests are represented by projectors in a Hilbert space.
    \item Compatible test are represented by commuting projectors.
    \item If $P$ and $Q$ are orthogonal projector then
        \begin{equation}
            \langle{P + Q}\rangle = \langle{P}\rangle + \langle{Q}\rangle
            \label{eq:QM_axiom3}
        \end{equation}
\end{enumerate}

For a non-contextual \HVT{} to exist, it has to be possible then to define a function $f(P)$ with values
$0$ or $1$ for every projector $P$ in $\Hil$.
Such a function cannot be written in the form \eqref{eq:gleason_prob}, required
by Gleason theorem for a probability distribution in $\Hil$, so for a \HVT{}, we
are forced to abandon axiom (3).
%talk about axiom 3

\subsection{Kochen-Specker theorem}
This fundamental incompatibility is better expressed by the Kochen-Specker
theorem.

This theorem affirms that in an Hilbert space $\Hil$ of dimension $d\ge 3$, there is no
function $h$ from the set of projectors of $\Hil$ to $\{0,1\}$ such that:
\begin{equation}
    \sum_i P_i = \I \implies \sum_i h(P_i) = 1 
    \label{eq:KS_function_prop}
\end{equation}

The proofs of this theorem are usually constructive: they present a set of
projectors and show how such a function $h$ cannot be defined.
The original proof used $117$ projectors in a three dimensional space, but
smaller sets exists.
For example this proof, due to A. Peres uses only 33 rays:
%cita
\begin{proof}
    The 33 rays in question are shown in figure and listed in the table below.
    We will show that is impossible to assign value $1$ or $0$ to these rays
    without violating \eqref{eq:KS_function_prop}.
    To better visualize the argument we will use the colors \emph{green} and
    \emph{red} to represent the values $1$ and $0$ respectively.

    \begin{center}
        \includegraphics[scale=1.5]{KS33cube.pdf} \\
    \end{center}
    \begin{center}
    \begin{tabular}{lrlrlr}
        \multicolumn{2}{l}{Peres KS set} \\
        \toprule
        $1$ & $(1, 0, 0)$ & $12$ & $(\sqrt{2}, 0, -1)$ & $23$ & $(-1, \sqrt{2}, 1)$ \\ 
        $2$ & $(0, 1, 0)$ & $13$ & $(\sqrt{2}, -1, 0)$  & $24$ & $(1, \sqrt{2}, -1)$ \\
        $3$ & $(0, 0, 1)$ & $14$ & $(\sqrt{2}, 1, 1)$  & $25$ & $(-1, \sqrt{2}, -1)$ \\
        $4$ & $(1, 1, 0)$ & $15$ & $(\sqrt{2}, -1, 1)$  & $26$ & $(0, 1,\sqrt{2})$ \\
        $5$ & $(0, 1, 1)$ & $16$ & $(\sqrt{2}, 1, -1)$  & $27$ & $(1, 0, \sqrt{2})$ \\
        $6$ & $(1, 0, 1)$ & $17$ & $(\sqrt{2}, -1, -1)$  & $28$ & $(0, -1,\sqrt{2})$ \\
        $7$ & $(-1, 1, 0)$ & $18$ & $(0, \sqrt{2}, 1)$  & $29$ & $(-1, 0,\sqrt{2})$ \\
        $8$ & $(0, -1, 1)$ & $19$ & $(1, \sqrt{2}, 0)$  & $30$ & $(1,1,\sqrt{2})$ \\
        $9$ & $(1, 0, -1)$ & $20$ & $(0, \sqrt{2}, -1)$  & $31$ & $(-1, 1, \sqrt{2})$ \\
        $10$ & $(\sqrt{2}, 0, 1)$ & $21$ & $(-1, \sqrt{2}, 0)$  & $32$ & $(1,-1, \sqrt{2})$ \\
        $11$ & $(\sqrt{2}, 1, 0)$ & $22$ & $(1, \sqrt{2}, 1)$  & $33$ & $(-1,-1, \sqrt{2})$ \\
    \end{tabular}   
    \end{center}
    As we can see the system is symmetric under the interchange or the reversal
    of the axes $x,y,z$.
    This means that without losing generality we can paint green any of
    the rays in the orthogonal triad $\{1,2,3\}$.
    Similarly in $\{1,5,8\}$  we can choose between $5$ or $8$ (reversal of the
    $y$ axis), between $6$ or $9$ in $\{2,6,9\}$ (reversal of the $x$ axis)
     and between $31$ or $32$ in $\{4,31,32\}$ (interchange of $x$ with $y$).

    If we choose for example to paint $3,5,6$ and $32$ green, the value of the other rays
    will be fixed by the procedure shown below: \\
    \begin{center}
    \begin{tabular}{lclcl}
        $\green{3}$  & $\implies$ & \multicolumn{3}{l}{\red{$1, 2, 4, 7,11, 13, 19, 21$}}\\
        $\green{5}$  & $\implies$ & \multicolumn{3}{l}{\red{$1, 8, 15, 16$}} \\
        $\green{6}$  & $\implies$ & \multicolumn{3}{l}{\red{$2, 9, 23, 24$}} \\
        $\green{32}$ & $\implies$ & \multicolumn{3}{l}{\red{$4,12,18,31$}} \\
        \midrule
        \red{$12,2$} & $\implies$ & \green{$27$} & $\implies$ & \red{$17$} \\
        \red{$8,17$} & $\implies$ & \green{$14$} & $\implies$ & \red{$29$} \\
        \red{$29,2$} & $\implies$ & \green{$10$} & $\implies$ & \red{$33$} \\
        \red{$33,7$} & $\implies$ & \green{$30$} & $\implies$ & \red{$20$} \\
        \red{$20,1$} & $\implies$ & \green{$26$} & $\implies$ & \red{$25$} \\
        \red{$25,8$} & $\implies$ & \green{$22$} & $\implies$ & \red{$28$} \\
    \end{tabular}
    \end{center}
    
    But with this coloring the rays $\{1,18,28\}$, which form an orthogonal
    triad, are all red: we have a contradiction, and since the initial assignments where
    completely general the theorem is proved.
\end{proof}

Clearly this means that an \HVT{}, that assigns definite values $0$ or $1$ to every
yes-no test regardless of the context, is fundamentally incompatible with \QM{}.
In fact every such assignment has to satisfy \eqref{eq:KS_function_prop}, since a
set of projectors that sums to identity represent a complete test, so the
assigned values must sum to unity.
But we have just that proved that this is not possible in general for $d \ge 3$.

Other kinds of proofs exists which relies on general observables (not just
projectors), like the Peres-Mermin square described in \ref{sec:PM_square}, or
on inequalities as shown in \ref{sec:YO13}.

\subsection{Experimental Test of KS theorem}
What we have shown is just that the \QM{} predictions are impossible to reproduce
with a classically behaving \HVT{}.
The next obvious step would be to see which way nature prefers by testing it
experimentally.
While it is possible to test the Kochen-Specker directly using the proof presented
in the last section (or other equivalent proofs), this approach is known to be
problematic.

One of the biggest problem is that every observable must be measured in more
than one, incompatible contexts. So one could not discard the possibility that
the results are influenced by the different measurement's procedure.
The inequalities, described in the next section, offer a better alternative for
experimental verification.

Another difficulty lies in the precision with which we can measure an
observable: there is always the chance that our measured observable could
differ slightly from the one we wanted to measure, this is called the \emph{finite
precision problem}.
Since it can be shown that it is possible to approximate a \acron{KS} set of rays with a
non \acron{KS} one, this is a serious problem in testing the \acron{KS}
theorem.
The way out of it is to create new versions of the \acron{KS} theorem
for imprecisely defined observables, often expressed as the already mentioned
inequalities.

The impossibility to precisely define an observable leads also to the
\emph{problem of compatibility}: we cannot assume that the observable are
compatible anymore, and as we have seen, this is a crucial assumption for
testing contextuality.
To deal with this problem, we need to add new terms to the classical bound in
inequalities, as will be described in the next section.

\section{Non-contextual inequalities}
In some systems, assuming classical
behavior implies that certain combinations of correlations between observables
are bounded by an inequality that is violated \QM{}.
This suggests another way to prove and to test the fundamental
incompatibility between \QM{} and non-contextual theories.

The most famous of these inequality is certainly the \emph{Bell theorem}, which
showed how \QM{} cannot be described by a local \HVT{}.  Since non-locality is a
particular form of contextuality, the same inequality can be used to settle the
question of the existence of non-contextual \HVT{}, and is described in section
\ref{sec:CHSH} (in the form of the \acron{CHSH} inequality).
%Scrivi nell'introduzione

Anyway, in the more general setting of contextuality we don't necessarily need two
subsystems or entanglement, as we showed in the KS theorem, a spin-1 system is enough.

It is useful to divide non-contextual inequalities into two classes:
\begin{itemize}
    \item \emph{State-dependent}, where the inequality is violated only when the
        system is in some particular state.
    \item \emph{State-independent}, where the inequality is always violated
        regardless of the state of the system.
\end{itemize}
Usually state-dependent tests involve less measurements, but obviously need a
fixed initial state, while state-independent does not.

As in section \ref{sec:joint_prob}, we can associate a graph to every
non-contextual inequality.
There are two ways to construct a graph for an inequality test: 
\begin{itemize}
    \item The \emph{Compatibility graph} is the graph constructed from the set of observables: in such a
        graph every observable is represented by a vertex and two vertex are
        connected if the corresponding observables are compatible.
    \item The \emph{Exclusivity graph} constructed from the single dichotomous exclusive events,
        where every vertex represent a yes-no measurement (a projector in the
        quantum formalism), and edges represent exclusivity (orthogonality in
        \QM{}).
\end{itemize}

In the next sections we will describe some of the most common non-contextual
inequalities.

\subsection{CHSH}
\label{sec:CHSH}
This inequality was first devised by Clauser, Horne, Shimony and Holt as an
alternative to the Bell inequality to test non-locality.
For this reason the system is composed by two 2-dimensional subsystems
(normally spin-$\frac{1}{2}$ particles) in
entanglement, that can be separated by a space-like distance.

The experimenters can measure the spin on their particles in two different directions.
The chosen observable are:
\begin{align}
    A_1 &= \sigma_x \otimes \I &
        B_1 &= \sigma_z \otimes \I \\
    A_2 &=  - \I \otimes \frac{\sigma_z + \sigma_x}{\sqrt{2}} 
        & B_2 &=  \I \otimes \frac{\sigma_z - \sigma_x}{\sqrt{2}} 
    \label{eq:CHSH_obs}
\end{align}
and the corresponding compatibility graph is shown in figure
\ref{fig:CHSH_obs_graph}.

Since the observables measured can takes only $\pm 1$ values if we suppose a classical,
non-contextual(or local) behavior we expect:
\begin{equation}
    \langle{(A_1 + B_1) A_2 + (B_1 - A_1) B_2}\rangle = 
    \langle{A_1A_2 + B_1A_2 + B_1B_2 - A_1B_2}\rangle \le 2
    \label{eq:CHSH_HVT}
\end{equation}

But if, say, we prepare the system in the entangled state
\begin{equation}
    \Ket{\psi} = \frac{\Ket{0}\otimes\Ket{1} -
    \Ket{1}\otimes\Ket{0}}{\sqrt{2}} 
    \label{eq:bell_state_psim}
\end{equation}
\QM{} predicts:
\begin{equation}
    \langle{A_1A_2}\rangle + \langle{B_1A_2}\rangle + \langle{B_1B_2}\rangle
    - \langle{A_1B_2}\rangle = 2\sqrt{2}
    \label{eq:CHSH_QM}
\end{equation}
so \eqref{eq:CHSH_HVT} is violated by quantum mechanics.
Notice that for the inequality to be violated the system has to be in a particular entangled
state, so this is an example of a state-dependent inequality.

We can also write \eqref{eq:CHSH_HVT} using directly probabilities for exclusive events:
\begin{multline}
    p(11| A_1 A_2) + p(-1-1| A_1 A_2) +
    p(11| B_1 B_2) + p(-1-1| B_1 B_2) +\\
    p(11| B_1 A_2) + p(-1-1| B_1 A_2) +
    p(-11| A_1 B_2) + p(1-1| A_1 B_2) \le 3
\end{multline}
where $p(ab|AB)$ represents the probability of having oucomes $a$ and $b$
measuring the properties $A$ and $B$.
The events involved are represented in the exclusivity graph \ref{fig:CHSH_proj_graph}.

It can be shown that in \QM{} the maximum value of this sum is $2+\sqrt{2}$, so
the inequality is again violated.
\begin{figure}[h]
    \begin{subfigure}{.3\textwidth}
        \centering
        \includegraphics[width=\textwidth]{CHSHog.pdf}
        \caption{Compatibility graph}
    \label{fig:CHSH_obs_graph}
    \end{subfigure}
    \begin{subfigure}{.7\textwidth}
        \centering
        \includegraphics[width=\textwidth]{CHSHpg.pdf}
        \caption{Exclusivity graph}
    \label{fig:CHSH_proj_graph}
    \end{subfigure}
    \caption{Graphs for the \acron{CHSH} inequality}
    \label{fig:CHSH_graphs}
\end{figure}
\subsection{KCBS}
A more recent example of a state-dependent non-contextual inequality is
called the \acron{KCBS} inequality, from Klyachko, Can, Binicioglu and Shumovsky.

Here we have a 3-dimensional system (like a spin-1 particle), and five
dichotomous observables which can take values $\{0,1\}$ and satisfy the
exclusivity relations shown in figure \ref{fig:KCBS_graph}.
\begin{figure}[h]
        \centering
        \includegraphics[width=0.4\textwidth]{KCBSog.pdf}
        \caption{Exclusivity graph for the \emph{KCBS} inequality}
    \label{fig:KCBS_graph}
\end{figure}
For example this can be implemented measuring $A_i = 1 - S_i^2$,
where the $S_i$ are the spin observables for a spin-1 particle in five
different direction.

Classically, since only 2 observables can be assigned the value
$1$ to satisfy the exclusivity graph, we have the inequality.
\begin{equation}
    \sum_{i=1}^5 a_i \le 2
    \label{eq:KCBS_HVT_1}
\end{equation}

This inequality is often presented in another form, using the
observables $B_i = 2 A_i - 1$ instead that takes the values $\{1,-1\}$, so that
\eqref{eq:KCBS_HVT_1} becomes:
\begin{equation}
    \langle{B_1B_2}\rangle + 
    \langle{B_2B_3}\rangle + 
    \langle{B_3B_4}\rangle + 
    \langle{B_4B_5}\rangle + 
    \langle{B_5B_1}\rangle \ge - 3 
    \label{eq:KCBS_HVT_2}
\end{equation}

In \QM{} the $A_i = \Ket{a_i}\Bra{a_i}$ are projectors in a 3-dimensional
Hilbert space, with orthogonality relations given by the graph
\ref{fig:KCBS_graph}.
For example we can use the rays:
\begin{align}
    \Ket{a_1} &= \left(1,0,\sqrt{\cos(\pi/5)}\right) \\
    \Ket{a_2} &= \left(-\cos(\pi/5),\sin(\pi/5),\sqrt{\cos(\pi/5)}\right)\\ 
    \Ket{a_3} &=\left(-\cos(\pi/5),-\sin(\pi/5),\sqrt{\cos(\pi/5)}\right)\\
    \Ket{a_4} &= \left(-\cos(2\pi/5),\sin(2\pi/5),\sqrt{\cos(\pi/5)}\right)\\
    \Ket{a_4} &=\left(-\cos(2\pi/5),-\sin(2\pi/5),\sqrt{\cos(\pi/5)}\right)\\
    \label{eq:KCBS_rays}
\end{align}
directed along the vertices of a pentagram.
If we measure them on the state $\Ket{\psi} = (1,0,0)$ we have
\begin{equation}
    \sum_{i=1}^5 \left|\Braket{a_i|\psi}\right|^2 = \sqrt{5} > 2
    \label{eq:KCBS_QM_1}
\end{equation}
which correspond to a violation of $5-4\sqrt{5}$ for the inequality
\eqref{eq:KCBS_HVT_2}.
\subsection{Peres-Mermin square}
\label{sec:PM_square}
This state-independent inequality has its origin in another proof of the
Kochen-Specker theorem.

Suppose to have a system similar to the one used for the \acron{CHSH}
inequality: two particles of spin-$\frac{1}{2}$, and consider the
observables:
\begin{align}
    A_{11}&=\I\otimes\sigma_z  &A_{12}&=\sigma_z\otimes\I &A_{13}&=\sigma_z\otimes\sigma_z \\
    A_{21}&=\sigma_x\otimes\I  &A_{22}&=\I\otimes\sigma_x &A_{23}&=\sigma_x\otimes\sigma_x \\
    A_{31}&=\sigma_x\otimes\sigma_z  &A_{32}&=\sigma_z\otimes\sigma_x &A_{33}&=\sigma_y\otimes\sigma_y \\
    \label{eq:PM_square}
\end{align}
The last element of every row is the product of the other two, and the same
is true for the columns, except for the last one since: 
\begin{equation}
    (\sigma_z\otimes\sigma_z) (\sigma_x\otimes\sigma_x) = - \sigma_y\otimes\sigma_y
    \label{eq:PM_lastcol}
\end{equation}

For this reason a fixed assignment of values $\{1,-1\}$ to every observable in
\eqref{eq:PM_square} is not possible, and this rules out the possibility of
having a non-contextual \HVT{} to explain those measures.

This can also be rephrased in the form of an inequality:
\begin{multline}
    \langle{A_{11}A_{12}A_{13}}\rangle + 
    \langle{A_{21}A_{22}A_{23}}\rangle + 
    \langle{A_{31}A_{32}A_{33}}\rangle +\\ +
    \langle{A_{11}A_{21}A_{31}}\rangle + 
    \langle{A_{12}A_{22}A_{32}}\rangle - 
    \langle{A_{13}A_{23}A_{33}}\rangle \le 4 
    \label{eq:PM_ineq}
\end{multline}
Which holds classically, since the third element in every correlation is the
product of the other two.

But in \QM{} we have:
\begin{multline}
    \langle{A_{11}A_{12}A_{13}}\rangle + 
    \langle{A_{21}A_{22}A_{23}}\rangle + 
    \langle{A_{31}A_{32}A_{33}}\rangle +\\ +
    \langle{A_{11}A_{21}A_{31}}\rangle + 
    \langle{A_{12}A_{22}A_{32}}\rangle - 
    \langle{A_{13}A_{23}A_{33}}\rangle = 6 
\end{multline}
so the inequality \eqref{eq:PM_ineq} is violated for every state of the system.

\subsection{Yu-Oh's 13 rays inequality(TODO)}
Todo.
\label{sec:YO13}

\subsection{Graph theoretical approach (TODO)}
The \emph{exclusivity} graph is a powerful instrument to
analyze inequalities.
In fact from the graph, we can easily deduce if the inequality is
violated by \QM{}, and how much is the violation, as is shown in the next theorems.

\begin{theorem}
    Given a graph $G$ associated with a non-contextual inequality $S$, the maximum value
    attainable by a non-contextual \HVT{} is equivalent to the independence number
    $\alpha(G)$ of the graph while the maximum for quantum theory is the Lovatzs number
    $\theta(G)$ of the graph, and
    \begin{equation}
        \alpha(G) \le \theta(G)
    \end{equation}
    \label{th:nc_graph_indilov}
\end{theorem}
\begin{proof}
    To do.
\end{proof}

\begin{theorem}
    The inequality associated with the graph $G$ is violated if and only if $G$
    contain as a subgraph a cycle of $n\ge5$ or its complement.
    \label{th:nc_graph_indinum}
\end{theorem}
\begin{proof}
    To do.
\end{proof}
\appendix
\section{Basic notions of graph theory(TODO)}
Todo
\label{sec:graph}
In this small section we define all the terms connected to graph theory used in
the text.
\begin{definition}[Graph]
    A graph is a collection of vertex and edges, where every edge connects two
    vertex.
\end{definition}
Kinds of graphs (cycles, complete graphs, etc\ldots)
\begin{definition}[Subgraph]
\end{definition}
\begin{definition}[Clique]
\end{definition}
\begin{definition}[Independence number]
\end{definition}
\begin{definition}[Lovatzs number]
\end{definition}
\end{document}
